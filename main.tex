\documentclass[12pt,a4paper]{article}

% Paquetes básicos
\usepackage[utf8]{inputenc}
\usepackage[spanish]{babel}
\usepackage{amsmath}
\usepackage{amsfonts}
\usepackage{amssymb}
\usepackage{graphicx}
\usepackage{hyperref}
\usepackage{enumitem}
\usepackage{geometry}

% Configuración de página
\geometry{margin=2.5cm}

% Configuración de hipervínculos
\hypersetup{
    colorlinks=true,
    linkcolor=blue,
    filecolor=magenta,      
    urlcolor=cyan,
    pdftitle={Sistema de apoyo a decisiones para la mitigación y respuesta a inundaciones pluviales en Montevideo},
}

% Información del documento
\title{Sistema de apoyo a decisiones para la mitigación y respuesta a inundaciones pluviales en Montevideo}
\author{Grupo 21 (IOGR 2025)}
\date{Noviembre 2025}

\begin{document}

\maketitle

\begin{center}
\textbf{Informe IOGR 2025}\\
\textbf{Gestión de Riesgos - Inundaciones}
\end{center}

\vspace{1cm}

\tableofcontents
\newpage

\section{Introducción}

Las inundaciones urbanas por lluvias o tormentas intensas constituyen uno de los peligros hidrometeorológicos con mayor recurrencia e impacto en Montevideo. El gobierno departamental mantiene una Red Hidrometeorológica (RHM) con polígonos donde se han identificado zonas de alta probabilidad de inundación.

Este informe se enfoca en la Gestión de Desastres e Investigación de Operaciones (IO) adaptado al peligro de inundaciones cuando hay lluvias o tormentas en un entorno urbano. El objetivo es diseñar un sistema de apoyo a decisiones (SAD) para: (i) reducir el riesgo esperado antes de la lluvia, (ii) organizar la respuesta durante el evento, y (iii) acelerar la recuperación posterior, con criterios de equidad territorial y eficiencia operativa.

\section{Marco de Gestión de Desastres aplicado a inundaciones}

\subsection{Mitigación}

Conjunto de acciones para disminuir la probabilidad y/o el impacto de la inundación. Proponemos:

\begin{itemize}
    \item \textbf{Infraestructura y micro-obras de drenaje:} limpieza y desobstrucción priorizada de bocas de tormenta, construcción de zanjas de retardo, cámaras de retención temporales, válvulas antirretorno, minidiques móviles en puntos críticos.
    
    \item \textbf{Ordenamiento territorial y normativas:} desincentivos a pavimentos impermeables en zonas RHM, exigencia de cisternas o techos verdes en nuevas obras.
    
    \item \textbf{Intervenciones basadas en la naturaleza:} jardines de lluvia, corredores verdes, aumento de infiltración en espacio público.
    
    \item \textbf{Educación y comunicación:} campañas de manejo de residuos para evitar obstrucciones; formación de brigadas barriales.
    
    \item \textbf{Señalización de calles} con riesgo alto de anegamiento y restricción de estacionamiento preventivo cuando se emiten alertas meteorológicas (ejemplo: coordinación con el Centro de Gestión de Movilidad).
    
    \item \textbf{Implementación de sistemas de aviso automatizados} (SMS o app) que notifiquen a los dueños de vehículos registrados en esas áreas.
    
    \item \textbf{Diseño de refugios} o zonas seguras de estacionamiento temporal en cotas altas.
\end{itemize}

\textbf{Decisiones típicas:} dónde invertir, cuántos recursos asignar, qué zonas priorizar según riesgo y vulnerabilidad.

\subsection{Preparación}

Acciones para estar listos ante pronósticos de tormenta:

\begin{itemize}
    \item \textbf{Preposicionamiento de equipos} (bombas, barreras, bolsas de arena) y asignación de cuadrillas con ventanas de tiempo basadas en el aviso meteorológico.
    
    \item \textbf{Planes de evacuación barrial} y cierres preventivos de calles de baja cota.
    
    \item \textbf{Protocolos de comunicación} con vecinos y servicios esenciales (salud, educación, tránsito).
    
    \item \textbf{Definición de protocolos de retiro preventivo de vehículos}, con coordinación entre la Intendencia y la Policía de Tránsito.
    
    \item \textbf{Integración con datos en tiempo real} del pronóstico y la RHM para activar cierres selectivos de calles.
\end{itemize}

\textbf{Decisiones típicas:} ubicación de depósitos/``kits'', número de cuadrillas por sector, rutas de llegada pre-planificadas.

\subsection{Respuesta}

Acciones durante el evento para preservar vidas, propiedad y continuidad de servicios:

\begin{itemize}
    \item \textbf{Despliegue de cuadrillas} a puntos RHM con mayor peligro inminente.
    
    \item \textbf{Cortes y desvíos viales} para disminuir el riesgo a las personas y permitir el acceso operativo.
    
    \item \textbf{Asistencia a población afectada} (agua, higiene, abrigo) mediante puntos de distribución temporales.
    
    \item \textbf{Modelos de ruteo} para grúas o cuadrillas que retiren autos antes del pico de lluvia.
    
    \item \textbf{Priorización de calles a cerrar} con base en niveles medidos por sensores RHM.
\end{itemize}

\textbf{Decisiones típicas:} ruteo dinámico con calles anegadas, asignación táctica de bombas/barreras, localización de puntos de distribución.

\subsection{Recuperación}

Acciones posteriores al evento para restaurar condiciones normales y fortalecer la resiliencia:

\begin{itemize}
    \item \textbf{Reparación y limpieza} de infraestructura de drenaje, levantamiento de daños.
    
    \item \textbf{Apoyo social} y restitución de bienes esenciales.
    
    \item \textbf{Evaluación post-evento} a modo de retrospectiva y accionables (lecciones aprendidas).
\end{itemize}

\textbf{Decisiones típicas:} programación de obras, inventarios de reposición, priorización territorial bajo restricciones presupuestales.

\section{Investigación de Operaciones aplicada}

\subsection{IO ``soft'' para estructuración y priorización}

Se propone un proceso participativo con SODA/SSM para entender el problema desde los actores (IM, SINAE, OSE, Juntas Locales, organizaciones vecinales, salud, educación, tránsito, empresas de servicios):

\begin{enumerate}
    \item \textbf{Entrevistas y mapas cognitivos} para identificar objetivos (p. ej., equidad territorial, minimizar población expuesta, proteger servicios críticos) y opciones (micro-obras, puntos de depósito, kits, rutas, protocolos).
    
    \item \textbf{CATWOE} para acordar el sistema relevante, restricciones y criterios.
    
    \item \textbf{Análisis multicriterio} (AHP/TOPSIS) para ponderar un Índice de Riesgo por zona RHM que combine peligro (historial de inundaciones), exposición (población, escuelas, hospitales), vulnerabilidad (NBI, edad), y criticidad (vías y servicios estratégicos).
\end{enumerate}

El resultado alimenta los modelos cuantitativos al proveer pesos de prioridad y umbrales de servicio.

\subsection{IO ``hard'' y formulaciones}

A continuación se detallan formulaciones base y su interpretación para inundaciones:

\subsubsection{(A) Cobertura máxima / Ubicación de depósitos (MCLP)}

\textbf{Decisión:} abrir hasta $(p)$ sitios (depósitos o bases operativas) para cubrir zonas RHM dentro de un tiempo objetivo $(T)$ minutos.

\textbf{Objetivo:} maximizar población/criticidad cubierta.

\textbf{Uso:} planificación de preparación.

\subsubsection{(B) p-mediana / Asignación}

\textbf{Decisión:} elegir $(p)$ bases minimizando el tiempo promedio de llegada a zonas RHM.

\textbf{Uso:} alternativa cuando importa el tiempo medio y no solo la cobertura.

\subsubsection{(C) Portafolio de micro-obras (mochila/knapsack con equidad)}

\textbf{Decisión:} seleccionar un conjunto de intervenciones de bajo costo bajo presupuesto $(B)$.

\textbf{Objetivo:} maximizar la reducción del riesgo esperado; restricciones de equidad por barrios vulnerables.

\subsubsection{(D) VRP con ventanas de tiempo (preposición y respuesta)}

\textbf{Decisión:} rutas de cuadrillas y camiones con horarios vinculados al pico de lluvia.

\textbf{Objetivo:} minimizar zonas críticas no atendidas antes del pico; segundo objetivo: costo/tiempo total.

\textbf{Extensiones:} escenarios estocásticos de intensidad de lluvia; calles anegadas.

\subsubsection{(E) Evacuación local y cortes (flujos + simulación de colas)}

\textbf{Decisión:} plan de cierres y desvíos que minimicen tiempo de vaciado de un polígono RHM, controlando colas en intersecciones.

\subsubsection{(F) LIRP post-evento (Location--Inventory--Routing)}

\textbf{Decisión:} dónde montar puntos de distribución (agua, higiene, abrigo), cuánto stock mantener y cómo reponer.

\textbf{Objetivo:} minimizar tiempo máximo sin atención de la población afectada y el costo logístico.

\section{Estudio de caso: zonas RHM de Montevideo}

\subsection{Datos y supuestos}

\begin{itemize}
    \item \textbf{Unidades de análisis:} centroides de polígonos RHM (ej.: Unión, Buceo, Malvín, Centro, Cerro).
    
    \item \textbf{Pesos por zona:} Índice de Riesgo (hazard, exposición, vulnerabilidad, criticidad) definido con AHP.
    
    \item \textbf{Red vial:} tiempos $(t_{ij})$ desde bases candidatas (CCZ, depósitos municipales, garajes) a cada zona RHM; penalizaciones por calles usualmente inundadas.
    
    \item \textbf{Recursos:} $(p)$ depósitos a habilitar, $(K)$ cuadrillas con capacidad de instalación de barreras/bombas por hora, flota de camiones.
    
    \item \textbf{Pronóstico:} ventana crítica de $(\pm)$ 2--3 h alrededor del pico de precipitación.
\end{itemize}

\subsection{Resultados de modelos}

\begin{enumerate}
    \item \textbf{MCLP/p-mediana:} selección de $(p)$ sitios que cubren $\geq X\%$ de población RHM en $\leq T$ min; mapa de cobertura y asignación de zonas.
    
    \item \textbf{Portafolio de micro-obras:} conjunto óptimo bajo $(B)$, con tabla de beneficio marginal por dólar y porcentaje en barrios priorizados.
    
    \item \textbf{VRP con ventanas de tiempo:} rutas para $(K)$ cuadrillas con llegada antes del pico a los $N$ puntos más críticos; tiempos previstos y holguras.
    
    \item \textbf{LIRP:} localización de 2--3 puntos de distribución y plan de reabastecimiento para 24--72 h.
\end{enumerate}

\subsection{Análisis de sensibilidad}

\begin{itemize}
    \item Cambios en $(p)$, $(T)$, $(B)$, severidad del evento, tiempos de viaje y capacidad de cuadrillas.
    
    \item Escenarios con calles inhabilitadas; robustez de las rutas y cobertura.
\end{itemize}

\section{Discusión y limitaciones}

\begin{itemize}
    \item \textbf{Datos:} la calidad del modelo depende de la resolución de RHM, inventario de infraestructura de drenaje y reportes históricos.
    
    \item \textbf{Incertidumbre:} intensidad y localización de lluvia varían; recomendamos escenarios y optimización robusta.
    
    \item \textbf{Gobernanza:} coordinación interinstitucional (IM, SINAE, OSE, Policía/Tránsito, Salud) y participación comunitaria son claves para la sostenibilidad y equidad.
    
    \item \textbf{Transferibilidad:} la metodología es replicable a otras ciudades con mapas de riesgo.
\end{itemize}

\section{Conclusiones}

\begin{enumerate}
    \item El marco de Mitigación--Preparación--Respuesta--Recuperación permite articular intervenciones de ingeniería, operativas y sociales de forma coherente para inundaciones urbanas.
    
    \item La IO habilita decisiones más informadas: (i) ubicación óptima de depósitos y puntos de distribución; (ii) portafolios de micro-obras costo--efectivas y equitativas; (iii) ruteo táctico con ventanas de tiempo para llegar antes del pico; (iv) logística de abastecimiento post-evento.
    
    \item En el estudio de caso, la combinación MCLP + VRP + portafolio muestra mejoras sustantivas en cobertura y tiempos de atención frente a asignaciones ad-hoc.
    
    \item Recomendamos institucionalizar el SAD con actualización continua de datos RHM, inventarios y lecciones aprendidas tras cada evento.
\end{enumerate}

\section{Bibliografía}

\begin{itemize}
    \item Altay, N., \& Green, W. G. (2006). OR/MS research in disaster operations management. \textit{European Journal of Operational Research}, 175(1), 475--493.
    
    \item Diapositivas en EVA del curso. \url{https://eva.fing.edu.uy/course/view.php?id=38}
    
    \item Ley 18621 Sistema Nacional de Emergencias Archivo
    
    \item Intendencia de Montevideo. Red Hidrometeorológica de Montevideo (RHM). \url{https://montevidata.montevideo.gub.uy/ambiental/red-hidrometeorologica-de-montevideo-rhm}
\end{itemize}

\end{document}
